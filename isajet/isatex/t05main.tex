\newpage
\section{Main Program\label{MAIN}}

      A main program for ISAJET for Linux/Unix is supplied in
\verb|isajet.tar|. To generate events and write them to disk, the user
should provide a main program which opens the files and then calls
subroutine ISAJET.  In the following sample, i,j,m,n are arbitrary unit
numbers.

      Main program for VMS:
\begin{verbatim}
      PROGRAM RUNJET
C
C          MAIN PROGRAM FOR ISAJET ON BNL VAX CLUSTER.
C
      OPEN(UNIT=i,FILE='$2$DUA14:[ISAJET.ISALIBRARY]DECAY.DAT',
     $STATUS='OLD',FORM='FORMATTED',READONLY)
      OPEN(UNIT=j,FILE='myjob.dat',STATUS='NEW',FORM='UNFORMATTED')
      OPEN(UNIT=m,FILE='myjob.par',STATUS='OLD',FORM='FORMATTED')
      OPEN(UNIT=n,FILE='myjob.lis',STATUS='NEW',FORM='FORMATTED')
C
      CALL ISAJET(+-i,+-j,m,n)
C
      STOP
      END
\end{verbatim}

      Main program for IBM (VM/CMS)
\begin{verbatim}
      PROGRAM RUNJET
C
C          MAIN PROGRAM FOR ISAJET ON IBM ASSUMING FILES HAVE BEEN
C          OPENED WITH FILEDEF.
C
      CALL ISAJET(+-i,+-j,m,n)
C
      STOP
      END
\end{verbatim}

      Main program for Unix; this is created by the standard 
\verb|Makefile|:
\begin{verbatim}
      PROGRAM RUNJET
C
C          Main program for ISAJET on Unix
C
      CHARACTER*60 FNAME
C
C          Open user files
      READ 1000, FNAME
1000  FORMAT(A)
      PRINT 1020, FNAME
1020  FORMAT(1X,'Data file      = ',A)
      OPEN(2,FILE=FNAME,STATUS='NEW',FORM='UNFORMATTED')
      READ 1000, FNAME
      PRINT 1030, FNAME
1030  FORMAT(1X,'Parameter file = ',A)
      OPEN(3,FILE=FNAME,STATUS='OLD',FORM='FORMATTED')
      READ 1000, FNAME
      PRINT 1040, FNAME
1040  FORMAT(1X,'Listing file   = ',A)
      OPEN(4,FILE=FNAME,STATUS='NEW',FORM='FORMATTED')
C          Open decay table
      READ 1000, FNAME
      OPEN(1,FILE=FNAME,STATUS='OLD',FORM='FORMATTED')
C
C          Run ISAJET
      CALL ISAJET(-1,2,3,4)
C
      STOP
      END
\end{verbatim}

      The arguments of ISAJET are tape numbers for files, all of which
should be opened by the main program.

      \verb|TAPEi|: Decay table (formatted). A positive sign prints
the decay table on the output listing. A negative sign suppress
printing of the decay table.

      \verb|TAPEj|: Output file for events (unformatted). A positive
sign writes out both resonances and stable particles. A negative sign
writes out only stable particles.

      \verb|TAPEm|: Commands as defined in Section 6 (formatted).

      \verb|TAPEn|: Output listing (formatted).

\noindent In the sample jobs in Section 3, TAPEm is the default
Fortran input, and TAPEn is the default Fortran output.

\subsection{Interactive Interface}

      To use the interactive interface, replace the call to ISAJET in
the above main program by
\begin{verbatim}
      CALL ISASET(+-i,+-j,m,n)
      CALL ISAJET(+-i,+-j,m,n)
\end{verbatim}
ISASET calls DIALOG, which prompts the user for possible commands,
does a limited amount of error checking, and writes a command file on
TAPEm. This command file is rewound for execution by ISAJET. A main
program is included in patch ISARUN to open the necessary files and to
call ISASET and ISAJET.

\subsection{User Control of Event Loop}

      If the user wishes to integrate ISAJET with another program and
have control over the event generation, he can call the driving
subroutines himself. The driving subroutines are:

      \verb|ISAINI(+-i,+-j,m,n)|: initialize ISAJET. The arguments are
the same as for subroutine ISAJET.

      \verb|ISABEG(IFL)|: begin a run. IFL is a return flag: IFL=0
for a good set of commands; IFL=1001 for a STOP; any other value means
an error.

      \verb|ISAEVT(I,OK,DONE)| generate event I. Logical flag OK
signifies a good event (almost always .TRUE.); logical flag DONE
signifies the end of a run.

      \verb|ISAEND|: end a run.

\noindent There are also subroutines provided to write standard ISAJET
records, or Zebra records if the Zebra option is selected:

      \verb|ISAWBG| to write a begin-of-run record, should be called
immediately after ISABEG

      \verb|ISAWEV| to write an event record, should be called
immediately after ISAEVT

      \verb|ISAWND| to write an end-of-run record, should be called
immediately after ISAEND

      The control of the event loop is somewhat complicated to
accomodate multiple evolution and fragmentation as described in
Section 11. Note in particular that after calling ISAEVT one should
process or write out the event only if OK=.TRUE. The check on the DONE
flag is essential if one is doing multiple evolution and
fragmentation. The following example indicates how events might be
generated, analyzed, and discarded (replace \verb|&| by \verb|+|
everywhere):
\begin{verbatim}
      PROGRAM SAMPLE
C
&SELF,IF=IMPNONE
      IMPLICIT NONE
&SELF
&CDE,ITAPES
&CDE,IDRUN
&CDE,PRIMAR
&CDE,ISLOOP
C
      INTEGER JTDKY,JTEVT,JTCOM,JTLIS,IFL,ILOOP
      LOGICAL OK,DONE
      SAVE ILOOP
C--------------------------------------------------------------------- 
C>         Open files as above
C>         Call user initialization
C--------------------------------------------------------------------- 
C
C          Initialize ISAJET
C
      CALL ISAINI(-i,0,m,n)
    1 IFL=0
      CALL ISABEG(IFL)
      IF(IFL.NE.0) STOP
C
C          Event loop
C
      ILOOP=0
  101 CONTINUE
        ILOOP=ILOOP+1
C          Generate one event - discard if .NOT.OK
        CALL ISAEVT(ILOOP,OK,DONE)
        IF(OK) THEN
C--------------------------------------------------------------------- 
C>         Call user analysis for event
C--------------------------------------------------------------------- 
        ENDIF
      IF(.NOT.DONE) GO TO 101
C
C          Calculate cross section and luminosity
C
      CALL ISAEND
C--------------------------------------------------------------------- 
C>         Call user summary
C--------------------------------------------------------------------- 
      GO TO 1
      END
\end{verbatim}

\subsection{Multiple Event Streams}

      It may be desirable to generate several different kinds of events
simultaneously to study pileup effects. While normally one would want
to superimpose minimum bias or low-pt jet events on a signal of
interest, other combinations might also be interesting. It would be
very inefficient to reinitialize ISAJET for each event. Therefore, a
pair of subroutines is provided to save and restore the context, i.e.
all of the initialization information, in an array. The syntax is
\begin{verbatim}
      CALL CTXOUT(NC,VC,MC)
      CALL CTXIN(NC,VC,MC)
\end{verbatim}
where VC is a real array of dimension MC and NC is the number of words
used, about 20000 in the standard case. If NC exceeds MC, a warning is
printed and the job is terminated. The use of these routines is
illustrated in the following example, which opens the files with names
read from the standard input and then superimposes on each event of
the signal sample three events of a pileup sample. It is assumed that
a large number of events is specified in the parameter file for the
pileup sample so that it does not terminate.
\begin{verbatim}
      PROGRAM SAMPLE
C
C          Example of generating two kinds of events.
C
      CHARACTER*60 FNAME
      REAL VC1(20000),VC2(20000)
      LOGICAL OK1,DONE1,OK2,DONE2
      INTEGER NC1,NC2,IFL,ILOOP,I2,ILOOP2
C
C          Open decay table
      READ 1000, FNAME
1000  FORMAT(A)
      OPEN(1,FILE=FNAME,STATUS='OLD',FORM='FORMATTED')
C          Open user files
      READ 1000, FNAME
      OPEN(3,FILE=FNAME,STATUS='OLD',FORM='FORMATTED')
      READ 1000, FNAME
      OPEN(4,FILE=FNAME,STATUS='NEW',FORM='FORMATTED')
      READ 1000,FNAME
      OPEN(13,FILE=FNAME,STATUS='OLD',FORM='FORMATTED')
      READ 1000,FNAME
      OPEN(14,FILE=FNAME,STATUS='NEW',FORM='FORMATTED')
C
C          Initialize ISAJET
      CALL ISAINI(-1,0,3,4)
      CALL CTXOUT(NC1,VC1,20000)
      CALL ISAINI(-1,0,13,14)
      IFL=0
      CALL ISABEG(IFL)
      IF(IFL.NE.0) STOP1
      CALL CTXOUT(NC2,VC2,20000)
      ILOOP2=0
      CALL user_initialization_routine
C
1     IFL=0
      CALL CTXIN(NC1,VC1,20000)
      CALL ISABEG(IFL)
      CALL CTXOUT(NC1,VC1,20000)
      IF(IFL.NE.0) GO TO 999
      ILOOP=0
C
C          Main event
C
101   CONTINUE
        ILOOP=ILOOP+1
        CALL CTXIN(NC1,VC1,20000)
        CALL ISAEVT(ILOOP,OK1,DONE1)
        CALL CTXOUT(NC1,VC1,20000)
        IF(.NOT.OK1) GO TO 101
        CALL user_analysis_routine
C
C          Pileup
C
        CALL CTXIN(NC2,VC2,20000)
        I2=0
201     CONTINUE
          ILOOP2=ILOOP2+1
          CALL ISAEVT(ILOOP2,OK2,DONE2)
          IF(OK2) I2=I2+1
          IF(DONE2) STOP2
          CALL user_analysis_routine
        IF(I2.LT.3) GO TO 201
        CALL CTXOUT(NC2,VC2,20000)
C
      IF(.NOT.DONE1) GO TO 101
C
C          Calculate cross section and luminosity
C
      CALL CTXIN(NC1,VC1,20000)
      CALL ISAEND
      GO TO 1
C
999   CALL CTXIN(NC2,VC2,20000)
      CALL ISAEND
      CALL user_termination_routine
      STOP
      END
\end{verbatim}
It is possible to superimpose arbitrary combinations of events,
including events of the same reaction type with different parameters.
In general the number of events would be selected randomly based on the
cross sections and the luminosity.

      At this time CTXOUT and CTXIN cannot be used with the Zebra
output routines.

\subsection{Main Programs for ISASUSY and ISASUGRA\label{sugrun}}

     In addition to the event generator, \verb|isajet.car| contains two
programs to calculate SUSY masses and decay modes: ISASUSY, which
accepts weak scale parameters, and ISASUGRA, which calculates the weak
scale parameters from those at some high scale. The main programs, SSRUN
and SUGRUN respectively, are included. They both prompt for interactive
input and then call the appropriate ISAJET subroutines. The output is
formatted and printed by SSPRT and SUGPRT respectively. Executables for
ISASUSY and ISASUGRA are built by the Unix \verb|Makefile| and VMS
\verb|isamake.com|. 

     It is fairly straightforward to modify these routines to scan SUSY
parameters, but given the variety of possible scans, no attempt has been
made to provide code for this. If only masses are needed, SSMSSM can be
modified to remove the calls to the routines that calculate branching
ratios.
